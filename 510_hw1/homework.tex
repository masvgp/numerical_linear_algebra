\documentclass{article}
\usepackage{concmath}
\usepackage{fancyhdr}
\usepackage{extramarks}
\usepackage{amsmath}
\usepackage{amsthm}
\usepackage{amsfonts}
\usepackage{tikz}
\usepackage[plain]{algorithm}
\usepackage{algpseudocode}

\usetikzlibrary{automata,positioning}

%
% Basic Document Settings
%

\topmargin=-0.45in
\evensidemargin=0in
\oddsidemargin=0in
\textwidth=6.5in
\textheight=9.0in
\headsep=0.25in

\linespread{1.1}

\pagestyle{fancy}
\lhead{\hmwkAuthorName}
\chead{\hmwkClass\ : \hmwkTitle}
%\rhead{\firstxmark}
\lfoot{\lastxmark}
\cfoot{\thepage}

\renewcommand\headrulewidth{0.4pt}
\renewcommand\footrulewidth{0.4pt}

\setlength\parindent{0pt}

%
% Create Problem Sections
%

\newcommand{\enterProblemHeader}[1]{
    \nobreak\extramarks{}{Problem \arabic{#1} continued on next page\ldots}\nobreak{}
    \nobreak\extramarks{Problem \arabic{#1} (continued)}{Problem \arabic{#1} continued on next page\ldots}\nobreak{}
}

\newcommand{\exitProblemHeader}[1]{
    \nobreak\extramarks{Problem \arabic{#1} (continued)}{Problem \arabic{#1} continued on next page\ldots}\nobreak{}
    \stepcounter{#1}
    \nobreak\extramarks{Problem \arabic{#1}}{}\nobreak{}
}

\setcounter{secnumdepth}{0}
\newcounter{partCounter}
\newcounter{homeworkProblemCounter}
\setcounter{homeworkProblemCounter}{1}
\nobreak\extramarks{Problem \arabic{homeworkProblemCounter}}{}\nobreak{}

%
% Homework Problem Environment
%
% This environment takes an optional argument. When given, it will adjust the
% problem counter. This is useful for when the problems given for your
% assignment aren't sequential. See the last 3 problems of this template for an
% example.
%
\newenvironment{homeworkProblem}[1][-1]{
    \ifnum#1>0
        \setcounter{homeworkProblemCounter}{#1}
    \fi
    \section{Problem \arabic{homeworkProblemCounter}}
    \setcounter{partCounter}{1}
    \enterProblemHeader{homeworkProblemCounter}
}{
    \exitProblemHeader{homeworkProblemCounter}
}

%
% Homework Details
%   - Title
%   - Due date
%   - Class
%   - Section/Time
%   - Instructor
%   - Author
%

\newcommand{\hmwkTitle}{Homework\ \#1}
\newcommand{\hmwkDueDate}{August 31, 2022}
\newcommand{\hmwkClass}{Numerical Linear Algebra}
\newcommand{\hmwkClassTime}{Section 1}
\newcommand{\hmwkClassInstructor}{Instructor: Professor Blake Barker\\}
\newcommand{\hmwkAuthorName}{\textbf{Michael Snyder}}

%
% Title Page
%

\title{
    \vspace{2in}
    \textmd{\textbf{\hmwkClass:\ \hmwkTitle}}\\
    \normalsize\vspace{0.1in}\small{Due\ on\ \hmwkDueDate\ at 10:00PM}\\
    \vspace{0.1in}\large{\textit{\hmwkClassInstructor\ \hmwkClassTime}}
    \vspace{3in}
}

\author{\hmwkAuthorName}
\date{}

\renewcommand{\part}[1]{\textbf{\large Part \Alph{partCounter}}\stepcounter{partCounter}\\}

%
% Various Helper Commands
%

% Useful for algorithms
\newcommand{\alg}[1]{\textsc{\bfseries \footnotesize #1}}

% For derivatives
\newcommand{\deriv}[1]{\frac{\mathrm{d}}{\mathrm{d}x} (#1)}

% For partial derivatives
\newcommand{\pderiv}[2]{\frac{\partial}{\partial #1} (#2)}

% Integral dx
\newcommand{\dx}{\mathrm{d}x}

% Alias for the Solution section header
\newcommand{\solution}{\textbf{\large Solution}}

% Probability commands: Expectation, Variance, Covariance, Bias
\newcommand{\E}{\mathrm{E}}
\newcommand{\Var}{\mathrm{Var}}
\newcommand{\Cov}{\mathrm{Cov}}
\newcommand{\Bias}{\mathrm{Bias}}

\begin{document}

\maketitle

\pagebreak
\section*{Problem 1.2}
%\begin{homeworkProblem}
    Suppose masses $m_1, m_2, m_3, m_4$ are located at positions $x_1, x_2, x_3, x_4$ in a line connected by springs with spring constants $k_{12}, k_{23}, k_{34}$ whose natural lengths of extension are $l_{12}, l_{23}, l_{34}$. Let $f_1, f_2, f_3, f_4$ denote the rightward forces on the masses, e.g., $f_1 = k_{12}(x_2 - x_1 - l_{12})$. \\
    
    \textbf{(a)} Write the 4x4 matrix equation relating the column vectors $f$ and $x$. Let $K$ denote the matrix in this equation.\\

    \textbf{Solution}
    \[
    \begin{pmatrix}
        f_1 \\
        f_2 \\
        f_3 \\
        f_4
    \end{pmatrix}
    =
    \begin{pmatrix}
        -k_{12} & k_{12} & 0 & 0\\
        k_{12} & -(k_{12} + k_{23}) & k_{23} & 0\\
        0 & k_{23} & -(k_{23} + k_{34}) & k_{34}\\
        0 & 0 & k_{34} & -k_{34}\\
    \end{pmatrix}
        +
    \begin{pmatrix}
        k_{12}l_{12} & 0 & 0 \\
        k_{12}l_{12} & -k_{23}l_{23} & 0\\
        0 & k_{23}l_{23} & -k_{34}l_{34}  \\
        0 & 0 & k_{34}k_{34}\\
    \end{pmatrix}   
    \]
    \\

    \textbf{(b)} What are the dimensions of the entries of $K$ in the physics sense.\\
    
    \textbf{Solution}
    The entries of $K$ are spring constants and have units of $N/m$ or $kg/s^2$.\\

    \textbf{(c)} What are the dimensions of $\det (K)$, again in the physics sense.\\
    
    \textbf{Solution}
    The dimensions of $\det (K)$ are $(N/m)^4$ or $(kg/s^2)^4$.\\

    \textbf{(d)} Suppose $K$ is given numerical values based on the units meters, kilograms, and seconds. Now the system is rewritten with a matrix $K'$ based on centimeters, grams, and seconds. What is the relationship of $K'$ to $K$? What is the relationship of $\det (K')$ to $\det(K)$?\\
    
    \textbf{Solution}
    Since $1kg = 1000g$, $K' = 1000 K$ and $\det(K') = 1000^4 \det(K)$.\\

%\end{homeworkProblem}

\pagebreak
\section*{Problem 1.3}
%\begin{homeworkProblem}
    Generalizing Example 1.3, we say that a square or rectangular matrix $R$ with entries $r_{ij}$ is \textit{upper-triangular} if $r_{ij} = 0$ for $i > j$. By considering what space is spanned by the first $n$ columns of $R$ and using $(1.8)$, show that if $R$ is a nonsingular $m \times m$ upper-triangular matrix, then $R^{-1}$ is also upper-triangular. \\

    \begin{proof}
    Suppose $R$ is a nonsingular $m \times m$ upper triangular matrix.\\
    
    We list two useful relationships:\\

    $\dagger$ The fact that $R$ is nonsingular implies that $r_{ii} \neq 0$ for $1 \leq i \leq m$.\\

    $\star$ The equation for $(1.8)$ is $e_j = \sum_{i=1}^m z_{ij} r_i$, where $z_{ij}$ is the $ij$-entry of $Z=R^{-1}$ and $r_i$ is the $i^{th}$ column of $R$.\\

    We will use $\dagger$ and $\star$ to show by induction that $Z = R^{-1}$ is upper-diagonal. We begin with two base cases, $R$ a $2 \times 2$ matrix and $R$ a $3 \times 3$ matrix. In the $2 \times 2$ case, we have
    \[
        RZ = \begin{pmatrix}
            r_{11} & r_{12} \\
            0 & r_{22} \\
        \end{pmatrix} 
        \begin{pmatrix}
            z_{11} & r_{12} \\
            z_{21} & r_{22} \\
        \end{pmatrix} 
        =
        \begin{pmatrix}
            1 & 0 \\
            0 & 1 \\

        \end{pmatrix} 
    \]

    Then using $\star$, we have

    \[e_1 =
        \begin{pmatrix}
            1 \\
            0 \\
        \end{pmatrix} 
        =
        z_{11}\begin{pmatrix}
            r_{11} \\
            0 \\
        \end{pmatrix}
        +
        z_{21} \begin{pmatrix}
            r_{12} \\
            r_{22} \\
        \end{pmatrix}
    \]
    and 
    \[e_2 =
        \begin{pmatrix}
            0 \\
            1 \\
        \end{pmatrix} 
        =
        z_{12}\begin{pmatrix}
            r_{11} \\
            0 \\
        \end{pmatrix}
        +
        z_{22} \begin{pmatrix}
            r_{12} \\
            r_{22} \\
        \end{pmatrix}
    \]
    As has already been asserted, by $\dagger$, $r_{22} \neq 0$. However, the second component of $e_1$, which we denote $(e_1)_2 = 0$. Thus, $z_{21} = 0$, and $Z_{2 \times 2}$ is upper diagonal.\\

    In the $3 \times 3$ case, we have
    \[
        RZ = \begin{pmatrix}
            r_{11} & r_{12} & r_{13} \\
            0 & r_{22} & r_{23} \\
            0 & 0 & r_{33}   \\
        \end{pmatrix} 
        \begin{pmatrix}
            z_{11} & z_{12} & z_{13} \\
            z_{21} & z_{22} & z_{23} \\
            z_{31} & z_{32} & z_{33}   \\
        \end{pmatrix} 
        =
        \begin{pmatrix}
            1 & 0 & 0 \\
            0 & 1 & 0 \\
            0 & 0 & 1   \\
        \end{pmatrix} 
    \]
        Again, using $\star$, we have

    \[e_1 =
        \begin{pmatrix}
            1 \\
            0 \\
            0
        \end{pmatrix} 
        =
        z_{11}\begin{pmatrix}
            r_{11} \\
            0 \\
            0
        \end{pmatrix}
        +
        z_{21} \begin{pmatrix}
            r_{12} \\
            r_{22} \\
            0
        \end{pmatrix}
        +
        z_{31} \begin{pmatrix}
            r_{13} \\
            r_{23} \\
            r_{33}
        \end{pmatrix}
    \]

    \[e_2 =
    \begin{pmatrix}
        0 \\
        1 \\
        0
    \end{pmatrix} 
    =
    z_{12}\begin{pmatrix}
        r_{11} \\
        0 \\
        0
    \end{pmatrix}
    +
    z_{22} \begin{pmatrix}
        r_{12} \\
        r_{22} \\
        0
    \end{pmatrix}
    +
    z_{32} \begin{pmatrix}
        r_{13} \\
        r_{23} \\
        r_{33}
    \end{pmatrix}
\]

\[e_3 =
\begin{pmatrix}
    0 \\
    0 \\
    1
\end{pmatrix} 
=
z_{11}\begin{pmatrix}
    r_{11} \\
    0 \\
    0
\end{pmatrix}
+
z_{23} \begin{pmatrix}
    r_{12} \\
    r_{22} \\
    0
\end{pmatrix}
+
z_{33} \begin{pmatrix}
    r_{13} \\
    r_{23} \\
    r_{33}
\end{pmatrix}
\]

From the $2 \times 2$ case, we already have $z_{21} = 0$. Using the same logic, we see that, since $(e_j)_i = 0$ for $i > j$ and $r_{ij} \neq 0$ for $j \geq i$, it is necessary that $z_{ij} = 0$ for $i > j$, $1 \leq i \leq 3$, $1 \leq j \leq 3$.\\

Applying induction, we now assume that $Z_{k \times k}$ is upper diagonal for $k > 3$, and that the result is verified for all matrices $Z$ less than $k \times k$. Then since $r_{kk} \neq 0$ by $\dagger$, we have $z_{kj} = 0$ for $k > j$. Otherwise, $(e_j)_k \neq 0$ for $k > j$. Therefore, the $m \times m$ matrix $Z = R^{-1}$ is an upper-diagonal matrix, as was to be shown.

\end{proof} 

%\end{homeworkProblem}

\pagebreak
\section*{Problem 1.4}
%\begin{homeworkProblem}
Let $f_1, \dots , f_2$ be a set of functions defined on the interval $[1, 8]$ with the property that for any numbers $d_1, \dots , d_8$, there exists a set of coefficients $c_1, \dots , c_8$ such that 
\[\sum_{j=1}^8 c_j f_j(i) = d_i, ~~~~~ i = 1, \dots , 8. \]

\textbf{(a)} Show by appealing to the theorems of this lecture that $d_1, \dots , d_8$ determine $c_1, \dots , c_8$ uniquely.

\begin{proof}
    Suppose to the contrary that $d_1, \dots , d_8$ do not uniquely determine $c_1, \dots , c_8$. Then $\exists$ $a_1, \dots , a_8$ such that \[d_i = \sum_{j=1}^8 a_j f_j(i) ~~~~ i = 1, \dots , 8.\] But this means \[0 = d_i - d_i = \sum_{i = 1}^8 (a_j - c_j)f_j(i), ~~~~ i = 1, \dots, 8.\] But this implies $a_j = c_j$, contradicting our assumption that the $d_i$ do not uniquely determine the $c_j$. Therefore, $d_1, \dots , d_8$ uniquely determines $c_1, \dots, c_8$.
\end{proof}

\textbf{(b)} Let $A$ be the $8 \times 8$ matrix representing the linear mapping from data $d_1, \dots , d_8$ to coefficients $c_1, \dots, c_8$. What is the $ij$-entry of $A^{-1}$?

\textbf{Solution:} The $ij$-entry of $A^{-1}$ is $f_j(i)$, since if $A$ maps data $d_1, \dots , d_8$ to coefficients $c_1, \dots, c_8$, then
\begin{align}
    \vec{c} &= A\vec{d}\\
    A^{-1}\vec{c} &= A^{-1}A\vec{d}\\
    A^{-1}\vec{c} &= \vec{d}
\end{align}
which means $A^{-1}$ is the matrix of functions $f_j(i)$.
%\end{homeworkProblem}

\pagebreak

%
% Non sequential homework problems
%

% Jump to problem 18


\end{document}
