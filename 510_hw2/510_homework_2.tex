\documentclass{article}
%\usepackage{concmath}
\usepackage{fancyhdr}
\usepackage{extramarks}
\usepackage{amsmath}
\usepackage{amsthm}
\usepackage{amsfonts}
\usepackage{tikz}
\usepackage[plain]{algorithm}
\usepackage{algpseudocode}

\usetikzlibrary{automata,positioning}

%
% Basic Document Settings
%

\topmargin=-0.45in
\evensidemargin=0in
\oddsidemargin=0in
\textwidth=6.5in
\textheight=9.0in
\headsep=0.25in

\linespread{1.1}

\pagestyle{fancy}
\lhead{\hmwkAuthorName}
\chead{\hmwkClass\ : \hmwkTitle}
%\rhead{\firstxmark}
\lfoot{\lastxmark}
\cfoot{\thepage}

\renewcommand\headrulewidth{0.4pt}
\renewcommand\footrulewidth{0.4pt}

\setlength\parindent{0pt}

%
% Create Problem Sections
%

\newcommand{\enterProblemHeader}[1]{
    \nobreak\extramarks{}{Problem \arabic{#1} continued on next page\ldots}\nobreak{}
    \nobreak\extramarks{Problem \arabic{#1} (continued)}{Problem \arabic{#1} continued on next page\ldots}\nobreak{}
}

\newcommand{\exitProblemHeader}[1]{
    \nobreak\extramarks{Problem \arabic{#1} (continued)}{Problem \arabic{#1} continued on next page\ldots}\nobreak{}
    \stepcounter{#1}
    \nobreak\extramarks{Problem \arabic{#1}}{}\nobreak{}
}

\setcounter{secnumdepth}{0}
\newcounter{partCounter}
\newcounter{homeworkProblemCounter}
\setcounter{homeworkProblemCounter}{1}
\nobreak\extramarks{Problem \arabic{homeworkProblemCounter}}{}\nobreak{}

%
% Homework Problem Environment
%
% This environment takes an optional argument. When given, it will adjust the
% problem counter. This is useful for when the problems given for your
% assignment aren't sequential. See the last 3 problems of this template for an
% example.
%
\newenvironment{homeworkProblem}[1][-1]{
    \ifnum#1>0
        \setcounter{homeworkProblemCounter}{#1}
    \fi
    \section{Problem \arabic{homeworkProblemCounter}}
    \setcounter{partCounter}{1}
    \enterProblemHeader{homeworkProblemCounter}
}{
    \exitProblemHeader{homeworkProblemCounter}
}

%
% Homework Details
%   - Title
%   - Due date
%   - Class
%   - Section/Time
%   - Instructor
%   - Author
%

\newcommand{\hmwkTitle}{Homework\ \#2}
\newcommand{\hmwkDueDate}{September 2, 2022}
\newcommand{\hmwkClass}{Numerical Linear Algebra}
\newcommand{\hmwkClassTime}{Section 1}
\newcommand{\hmwkClassInstructor}{Instructor: Professor Blake Barker\\}
\newcommand{\hmwkAuthorName}{\textbf{Michael Snyder}}

%
% Title Page
%

\title{
    \vspace{2in}
    \textmd{\textbf{\hmwkClass:\ \hmwkTitle}}\\
    \normalsize\vspace{0.1in}\small{Due\ on\ \hmwkDueDate\ at 10:00PM}\\
    \vspace{0.1in}\large{\textit{\hmwkClassInstructor\ \hmwkClassTime}}
    \vspace{3in}
}

\author{\hmwkAuthorName}
\date{}

\renewcommand{\part}[1]{\textbf{\large Part \Alph{partCounter}}\stepcounter{partCounter}\\}

%
% Various Helper Commands
%

% Useful for algorithms
\newcommand{\alg}[1]{\textsc{\bfseries \footnotesize #1}}

% For derivatives
\newcommand{\deriv}[1]{\frac{\mathrm{d}}{\mathrm{d}x} (#1)}

% For partial derivatives
\newcommand{\pderiv}[2]{\frac{\partial}{\partial #1} (#2)}

% Integral dx
\newcommand{\dx}{\mathrm{d}x}

% Alias for the Solution section header
\newcommand{\solution}{\textbf{\large Solution}}

% Probability commands: Expectation, Variance, Covariance, Bias
\newcommand{\E}{\mathrm{E}}
\newcommand{\Var}{\mathrm{Var}}
\newcommand{\Cov}{\mathrm{Cov}}
\newcommand{\Bias}{\mathrm{Bias}}

\begin{document}

\maketitle

\pagebreak
\section*{Problem 2.2}
The Pythagorean Theorem asserts that for a set of $n$ orthogonal vectors $\{x_i\}$,
\begin{equation}
    \begin{Vmatrix}\displaystyle
        \sum_{i=1}^n x_i
    \end{Vmatrix}^2 = \sum_{i = 1}^n \begin{Vmatrix}
        x_i
    \end{Vmatrix}^2\\
\end{equation}
 

\textbf{(a)} Prove this in the case $n=2$ by an explicit computation of $\begin{Vmatrix}
    x_1 + x_2
\end{Vmatrix}^2$.

\begin{proof}
    
We begin by noting that by orthogonality, $x_i^* x_j = 0$ for $i \neq j$. Thus,
\begin{align*}
    \begin{Vmatrix}\displaystyle
        \sum_{i=1}^n x_i
    \end{Vmatrix}^2 &=     \begin{Vmatrix}
    x_1 + x_2
\end{Vmatrix}^2\\
 &= (x_1 + x_2)^*(x_1 + x_2)\\
 &= x_1^*x_1 + x_1^*x_2 + x_2^* x_1 + x_2^*x_2\\
 &= 1 + 0 + 0 + 1\\
 &= 1 + 1\\
 &= x_1^*x_1 + x_2^*x_2\\
 &= \begin{Vmatrix}
    x_1
 \end{Vmatrix}^2 + \begin{Vmatrix}
    x_2
 \end{Vmatrix}^2\\
 &= \sum_{i=1}^n \begin{Vmatrix}
    x_1
 \end{Vmatrix}^2.
\end{align*}
\end{proof} 
\textbf{(b)} Show that this computation also establishes the general case, by induction.
\begin{proof}
    The base case for induction has been shown in Part $(a)$. Note orthogonality is used to anhilate terms $x_i^*x_j$ for $i \neq j$ in $(3)$ below. Now, suppose Equation (1) equation holds for $k > 2$. We now show, by induction, that this implies Equation (1) also holds for $k+1$:
    \begin{align}
        \begin{Vmatrix}
            \sum_{i=1}^{k+1}x_1
        \end{Vmatrix}^2 &= (x_1 + \cdots + x_k + x_{k+1})^*(x_1 + \cdots + x_k + x_{k+1})\\
        &= \sum_{i=1}^k \begin{Vmatrix}
            x_i
        \end{Vmatrix}^2 + \sum_{i=1}^kx_{k+1}^*x_i + x_{k+1}^*x_{k+1}\\
        &= \sum_{i=1}^{k+1}\begin{Vmatrix}
            x_i
        \end{Vmatrix}^2.
    \end{align}
\end{proof}

\section*{Problem 2.5}
Let $S\in \mathbb{C}^{m \times m}$ be \textit{skew-hermitian}, i.e., $S^* = -S$.\\

\textbf{(a)} Show by using Exercise 2.1 that the eigenvalues of $S$ are pure imaginary.
\begin{proof}
    We show this directly without using Exerciese 2.1. Suppose $S\in \mathbb{C}^{m \times m}$ be \textit{skew-hermitian}. Then 
    \begin{equation}
        S^* = -S.    
    \end{equation}
    Futher, suppose that $\lambda$ is an eigenvalue of $S$ with associated eigenvector $x$. Then, using Equation (5), we have
    \begin{align*}
        S^* &= -S\\
        S^*x &= -Sx\\
        S^*x &= -\lambda x\\
        x^*Sx &= -\overline{\lambda}x^*x\\
        x^*\lambda x &= -\overline{\lambda}x^*x\\
        \lambda \begin{Vmatrix}
            x
        \end{Vmatrix} &= -\overline{\lambda}\begin{Vmatrix}
            x
        \end{Vmatrix}\\
        \begin{Vmatrix}
            x
        \end{Vmatrix}(\lambda + \overline{\lambda}) &= 0
    \end{align*}
    Since $x \neq 0$, it must be that $\lambda = -\overline{\lambda}$. If $\lambda = a + bi$, then we have
    \begin{align*}
        a+bi &= -(a - bi)\\
        a + bi &= -a + bi\\
    \end{align*}
    But this implies $a=0$. Therefore, $\lambda$ is pure imaginary, as was to be shown.
\end{proof}

\textbf{(b)} Show that $I-S$ is non-singular.
\begin{proof}
    Suppose to the contrary that $I-S$ is singular. Then there exists $x \in C^m$ such that $x \neq 0$, but $(I-S)x = 0$. Using this fact, we have
    \begin{align*}
        (I-S)x &= 0\\
        Ix - Sx &= 0\\
        x &= Sx\\
    \end{align*}
    But this means $\lambda = 1$ is an eigenvalue of $S$. By Part (a), this is a contradiction since all eigenvalues of $S$ are pure imaginary. Therefore, $I-S$ is non-singular.
\end{proof}

\textbf{(c)} Show that the matrix $Q = (I-S)^{-1}(I+S)$, known as the \textit{Cayley transform} of $S$, is unitary. 
\begin{proof}
    In this proof, we use the fact from Part $(a)$ that $S^* = -S$. Note that by definition, a matrix $A$ is unitary if $A^* = A^{-1}$. Thus, if $Q$ is unitary, then $QQ^* = QQ^{-1} = I$. We prove this fact as follows.
    \begin{align*}
        QQ^* &= (I-S)^{-1}(I+S)[(I-S)^{-1}(I+S)]^*\\
        &= (I-S)^{-1}(I+S)(I+S)^*(I-S)^{-*}\\
        &= (I-S)^{-1}(I+S)(I-S)(I+S)^{-1}\\
        &= (I-S)^{-1}(I-S + S - S^2)(I+S)^{-1}\\
        &= (I-S)^{-1}(-S + I)(S + I)(I+S)^{-1}\\
        &= (I-S)^{-1}(I-S)(I+S)(I+S)^{-1}\\
        &= II\\
        &= I.
    \end{align*}
\end{proof}

\section*{Problem 2.7}
A \textit{Hadamard matrix} is a matrix whose entries are all $\pm 1$ and whose transpose is equal to its inverse times a constant factor. It is know that if $A$ is a Hadamard matrix of dimension $m > 2$, then $m$ is a multiple of $4$. It is not know, however, whether there is a Hadamard matrix for every such $m$, though examples have been found for all cases $m \leq 424$.\\

Show that the following recursive description provides a Hadamard matrix of each dimension $m = 2^k, k = 0, 1, 2, \dots$:
\[ H_0 = \begin{bmatrix}
    1
\end{bmatrix}, ~~~ H_{k+1} = \begin{bmatrix}
    H_k & H_k\\
    H_k & -H_k\\
\end{bmatrix}. \]

\begin{proof}
    We use repeatedly the property that, if $A$ is a Hadamard matrix, then $A^T = aA^{-1}$ for some constant $a$. We prove the hypothesis by induction. As a base case, consider $k=0$.
    \[H_1 = \begin{bmatrix}
        H_0 & H_0\\
        H_0 & -H_0\\
    \end{bmatrix} = \begin{bmatrix}
        1 & 1\\
        1 & -1\\
    \end{bmatrix}.\]
    Here all entries are $\pm 1$ as required. Moreover, we note that the columns of $H_1$ are linearly independent, which can be verified by taking the dot product of the columns to obtain $1 - 1 = 0$. Factoring out a constant $a = 2$ so that the columns are of unit length yields a constant times the unitary matrix $aQ = H_1$. By defnition, $Q$ unitary means $Q^* = Q^{-1}$. Since Q contains only real values, we have $Q^T = Q^{-1}$. Hence
    \[aQ^T = H_1^T \Rightarrow aQ^{-1} = H_1^T.\] But this implies that $aQ^{-1} = aH_1^{-1}$. 
    
    % we compute the inverse of $H_1$ which yields
    % \[H_1^{-1} = -\frac{1}{2H_0}\begin{bmatrix}
    %     -H_0 & -H_0\\
    %     -H_0 & H_0\\
    % \end{bmatrix} \]
    % We also note that since $H_1$ is symmetric, $H_1^T = H_1$. Thus, the computation of $H_1^{-1}$ implies that
    % \begin{equation}
    %     H_1^T = 2H_0H_1^{-1} = 2H_1^{-1}. 
    % \end{equation}

    % In the case that $k=1$, we have
    % \[H_2 = \begin{bmatrix}
    %     H_1 & H_1\\
    %     H_1 & -H_1\\
    % \end{bmatrix}.\]
    % Again $H_2$ symmetric implies $H_2^T = H_2$. 

    Now assume that the recurrence has been verified to produce Hadamard matrices for all $m\times m$ matrices for $m=2^{k-1}$ for $k > 1$. Then considering $k$ we have by way of the recurrence,
    \[H_{k+1} = \begin{bmatrix}
        H_k & H_k\\
        H_k & -H_k\\
    \end{bmatrix} 
    \]
    Since $H_{k+1}$ is composed of linearly independent submatrices which are linearly independent, the columns of $H_{k+1}$ are linearly independent.
    Thus we can apply the same procedure to show that $H_{k+1}^T = aH_{k+1}^{-1}$. That is, we can factor out a constant to obtain a unitary matrix whose transpose is the inverse to obtain $aQ^{-1} = aH_1^{-1} = H_1^T$. This fact combined with the fact that all entries of $H_{k+1}$ are $\pm 1$, means that $H_{k+1}$ is a Hadamard matrix of size $m=2^{k+1}$. Therefore, for dimensions $m=2^k, k = 0, 1, 2, \dots$, the recurrence produces Hadamard matrix.


\end{proof}
\end{document}
