\documentclass{article}
%\usepackage{concmath}
\usepackage{fancyhdr}
\usepackage{extramarks}
\usepackage{amsmath}
\usepackage{amsthm}
\usepackage{amsfonts}
\usepackage{tikz}
\usepackage[plain]{algorithm}
\usepackage{algpseudocode}

\usetikzlibrary{automata,positioning}

%
% Basic Document Settings
%

\topmargin=-0.45in
\evensidemargin=0in
\oddsidemargin=0in
\textwidth=6.5in
\textheight=9.0in
\headsep=0.25in

\linespread{1.1}

\pagestyle{fancy}
\lhead{\hmwkAuthorName}
\chead{\hmwkClass\ : \hmwkTitle}
%\rhead{\firstxmark}
\lfoot{\lastxmark}
\cfoot{\thepage}

\renewcommand\headrulewidth{0.4pt}
\renewcommand\footrulewidth{0.4pt}

\setlength\parindent{0pt}

%
% Create Problem Sections
%

\newcommand{\enterProblemHeader}[1]{
    \nobreak\extramarks{}{Problem \arabic{#1} continued on next page\ldots}\nobreak{}
    \nobreak\extramarks{Problem \arabic{#1} (continued)}{Problem \arabic{#1} continued on next page\ldots}\nobreak{}
}

\newcommand{\exitProblemHeader}[1]{
    \nobreak\extramarks{Problem \arabic{#1} (continued)}{Problem \arabic{#1} continued on next page\ldots}\nobreak{}
    \stepcounter{#1}
    \nobreak\extramarks{Problem \arabic{#1}}{}\nobreak{}
}

\setcounter{secnumdepth}{0}
\newcounter{partCounter}
\newcounter{homeworkProblemCounter}
\setcounter{homeworkProblemCounter}{1}
\nobreak\extramarks{Problem \arabic{homeworkProblemCounter}}{}\nobreak{}

%
% Homework Problem Environment
%
% This environment takes an optional argument. When given, it will adjust the
% problem counter. This is useful for when the problems given for your
% assignment aren't sequential. See the last 3 problems of this template for an
% example.
%
\newenvironment{homeworkProblem}[1][-1]{
    \ifnum#1>0
        \setcounter{homeworkProblemCounter}{#1}
    \fi
    \section{Problem \arabic{homeworkProblemCounter}}
    \setcounter{partCounter}{1}
    \enterProblemHeader{homeworkProblemCounter}
}{
    \exitProblemHeader{homeworkProblemCounter}
}

%
% Homework Details
%   - Title
%   - Due date
%   - Class
%   - Section/Time
%   - Instructor
%   - Author
%

\newcommand{\hmwkTitle}{Homework\ \#3}
\newcommand{\hmwkDueDate}{September 7, 2022}
\newcommand{\hmwkClass}{Numerical Linear Algebra}
\newcommand{\hmwkClassTime}{Section 1}
\newcommand{\hmwkClassInstructor}{Instructor: Professor Blake Barker\\}
\newcommand{\hmwkAuthorName}{\textbf{Michael Snyder}}

%
% Title Page
%

\title{
    \vspace{2in}
    \textmd{\textbf{\hmwkClass:\ \hmwkTitle}}\\
    \normalsize\vspace{0.1in}\small{Due\ on\ \hmwkDueDate\ at 10:00PM}\\
    \vspace{0.1in}\large{\textit{\hmwkClassInstructor\ \hmwkClassTime}}
    \vspace{3in}
}

\author{\hmwkAuthorName}
\date{}

\renewcommand{\part}[1]{\textbf{\large Part \Alph{partCounter}}\stepcounter{partCounter}\\}

%
% Various Helper Commands
%

% Useful for algorithms
\newcommand{\alg}[1]{\textsc{\bfseries \footnotesize #1}}

% For derivatives
\newcommand{\deriv}[1]{\frac{\mathrm{d}}{\mathrm{d}x} (#1)}

% For partial derivatives
\newcommand{\pderiv}[2]{\frac{\partial}{\partial #1} (#2)}

% Integral dx
\newcommand{\dx}{\mathrm{d}x}

% Alias for the Solution section header
\newcommand{\solution}{\textbf{\large Solution}}

% Probability commands: Expectation, Variance, Covariance, Bias
\newcommand{\E}{\mathrm{E}}
\newcommand{\Var}{\mathrm{Var}}
\newcommand{\Cov}{\mathrm{Cov}}
\newcommand{\Bias}{\mathrm{Bias}}

\begin{document}

\maketitle

\pagebreak
\section*{Problem 3.3}
Vector and matrix $p$-norms are related by various inequalities, often involving dimensions $m$ or $n$. For each of the following, verify the inequality and give an example of a nonzero vector or matrix (for general $m, n$) for which equality is achieved. In this problem $x$ is an $m$-vector and $A$ is an $m\times n$ matrix.

\textbf{(a)}
$\begin{Vmatrix}
        x
    \end{Vmatrix}_\infty  \leq \begin{Vmatrix}
        x
    \end{Vmatrix}_2$
    \begin{proof}
        \begin{align}
            \begin{Vmatrix} x \end{Vmatrix}_\infty &= \max_{1 \leq i \leq m} \lvert x_i \rvert\\
            &= \max_{1 \leq i \leq m} \left( \lvert x_i \rvert^2 \right)^{1/2}\\
            &\leq \left( \sum_{i = 1}^m \lvert x_i \rvert^2 \right)^{1/2}\\
            &= \begin{Vmatrix}
                x
            \end{Vmatrix}_2
        \end{align}
        Note that from (2) to (3) we use the fact that adding all elements smaller than the $\max$ $x_i$ to $x_i$ is larger, hence the inequality.
    \end{proof}
    
    As an example where the equality holds, consider $e_1$. In this case, \[ \begin{Vmatrix} e_1 \end{Vmatrix}_\infty = 1 = \begin{Vmatrix} e_1 \end{Vmatrix}_2 \].

    \textbf{(b)} $\begin{Vmatrix}
        x
    \end{Vmatrix}_2 \leq \sqrt{m}\begin{Vmatrix}
        x
    \end{Vmatrix}_2$
    \begin{proof}
        \begin{align}
            \begin{Vmatrix}
                x
            \end{Vmatrix}_2
            &= \left( \sum_{i = 1}^m \lvert x_i \rvert^2 \right)^{1/2}\\
            &\leq \left( m \cdot \max_{1 \leq i \leq m} \lvert x_i \rvert^2  \right)^{1/2}\\
            &= \sqrt{m}\max_{1 \leq i \leq m} \lvert x_i \rvert\\
            &= \sqrt{m}\begin{Vmatrix}
                x
            \end{Vmatrix}_\infty
        \end{align}
        Note: The expression in (6) takes $m$ of the $\max$ $x_i$, hence is larger than the sum of the $x_i$.
    \end{proof}

    As an example, consider the vector containing all $1$s. In this case, \[ \begin{Vmatrix}
        x
    \end{Vmatrix}_2 = \sqrt{m} = \sqrt{m} \cdot 1 = \sqrt{m}\begin{Vmatrix}
        x
    \end{Vmatrix}_\infty.\]

\textbf{(c)} 
$
\begin{Vmatrix}
    A
\end{Vmatrix}_\infty 
\leq 
\sqrt{n} 
\begin{Vmatrix}
    A
\end{Vmatrix}_2
$

\begin{proof}
    We will use $(3.6)$ from page 19 of the text to prove the claim; the definition for induced matrix norms, given here.
    \[ 
    \begin{Vmatrix}
        A
    \end{Vmatrix}_{(m, n)}
    =
    \sup_{x \neq 0}
    \frac{
    \begin{Vmatrix}
        Ax
    \end{Vmatrix}_{(m)}   
    }{
    \begin{Vmatrix}
        x
    \end{Vmatrix}_{(n)} 
    } 
    \]
    
    By Part (a) 
    \[
    \begin{Vmatrix}
        Ax
    \end{Vmatrix}_{\infty}
    \leq
    \begin{Vmatrix}
        Ax
    \end{Vmatrix}_{2}.
    \]
    By Part (b),

    \[
    \begin{Vmatrix}
        x
    \end{Vmatrix}_{2}
    \leq
    \sqrt{n}
    \begin{Vmatrix}
        x
    \end{Vmatrix}_{\infty}. 
    \]
    This implies 
    \[
    \frac{
        1
    }{
        \begin{Vmatrix}
            x
        \end{Vmatrix}_{\infty}
    } 
    \leq 
    \frac{
        \sqrt{n}
    }{
        \begin{Vmatrix}
            x
        \end{Vmatrix}_{2}
    }.
    \]
    Putting the above relationships together, we have 
    
    \[
        \begin{Vmatrix}
            A
        \end{Vmatrix}_{\infty}
        =
        \sup_{x \neq 0}
        \frac{
            \begin{Vmatrix}
                Ax
            \end{Vmatrix}_{\infty}
        }{
            \begin{Vmatrix}
                x
            \end{Vmatrix}_{\infty}
        } 
        \leq 
        \sup_{x \neq 0}
        \frac{
            \sqrt{n}
            \begin{Vmatrix}
                Ax
            \end{Vmatrix}_{2}
        }{
            \begin{Vmatrix}
                x
            \end{Vmatrix}_{2}
        } 
        =
        \sqrt{n}
        \begin{Vmatrix}
            A
        \end{Vmatrix}_{2}
    \]
        Therefore, $
        \begin{Vmatrix}
            A
        \end{Vmatrix}_{\infty} \leq \sqrt{n}
        \begin{Vmatrix}
            A
        \end{Vmatrix}_{2}$

\end{proof}

As an example, Consider 
$A = 
\begin{bmatrix}
    1/\sqrt{2} & 1/\sqrt{2}\\
    0 & 0\\
\end{bmatrix} $
 and
 $x = \begin{bmatrix}
    1/\sqrt{2}\\
    1/\sqrt{2}\\
 \end{bmatrix}$. Computing the $\infty$-norm for $Ax$ and $x$ yields:
 \[ 
    \begin{Vmatrix}
        x
    \end{Vmatrix}_\infty
    =
    \begin{Vmatrix}
        \begin{bmatrix}
            1/\sqrt{2}\\
            1/\sqrt{2}\\
        \end{bmatrix}
    \end{Vmatrix}_\infty
    = \max \{1/\sqrt{2}, 1/\sqrt{2}\}
    = 1/\sqrt{2},
 \]
 and
 \[
    \begin{Vmatrix}
        Ax
    \end{Vmatrix}_\infty
    =
    \begin{Vmatrix}
        \begin{bmatrix}
            1/\sqrt{2} & 1/\sqrt{2}\\
            0 & 0\\
        \end{bmatrix}
        \begin{bmatrix}
            1/\sqrt{2}\\
            1/\sqrt{2}\\
        \end{bmatrix}
    \end{Vmatrix}_\infty
    =
    \begin{Vmatrix}
        \begin{bmatrix}
            1\\
            0\\
        \end{bmatrix}
    \end{Vmatrix}_\infty
    =
    1
 \]
 Thus, 
 \[
    \begin{Vmatrix}
        A
    \end{Vmatrix}_\infty
    =
    \frac{
            \begin{Vmatrix}
                Ax
            \end{Vmatrix}_{\infty}
        }{
            \begin{Vmatrix}
                x
            \end{Vmatrix}_{\infty}
        } 
    =
    \frac{1}{1/\sqrt{2}} = \sqrt{2}.
\]
Computing the $2$-norm for $Ax$ and $x$ yields:
 \[ 
    \begin{Vmatrix}
        x
    \end{Vmatrix}_2
    =
    \begin{Vmatrix}
        \begin{bmatrix}
            1/\sqrt{2}\\
            1/\sqrt{2}\\
        \end{bmatrix}
    \end{Vmatrix}_2
    = \sqrt{\left(1/\sqrt{2}\right)^2 + \left(1/\sqrt{2}\right)^2}
    = 1,
 \]
 and
 \[
    \begin{Vmatrix}
        Ax
    \end{Vmatrix}_2
    =
    \begin{Vmatrix}
        \begin{bmatrix}
            1/\sqrt{2} & 1/\sqrt{2}\\
            0 & 0\\
        \end{bmatrix}
        \begin{bmatrix}
            1/\sqrt{2}\\
            1/\sqrt{2}\\
        \end{bmatrix}
    \end{Vmatrix}_2
    =
    \begin{Vmatrix}
        \begin{bmatrix}
            1\\
            0\\
        \end{bmatrix}
    \end{Vmatrix}_2
    =
    1
 \]
 Thus, 
 \[ \sqrt{2}
    \begin{Vmatrix}
        A
    \end{Vmatrix}_2
    =
    \sqrt{2}
    \frac{
            \begin{Vmatrix}
                Ax
            \end{Vmatrix}_{2}
        }{
            \begin{Vmatrix}
                x
            \end{Vmatrix}_{2}
        } 
    =
    \sqrt{2}\cdot \frac{1}{1} = \sqrt{2}.
\]
Putting the computations together, we have 
\[
    \begin{Vmatrix}
        A
    \end{Vmatrix}_\infty
    =
    \sqrt{2}
    =
    \sqrt{2}
    \begin{Vmatrix}
        A
    \end{Vmatrix}_2
\]


\textbf{(d)}         
$\begin{Vmatrix}
    A
\end{Vmatrix}_{2} \leq \sqrt{m}
\begin{Vmatrix}
    A
\end{Vmatrix}_{\infty}$
\begin{proof}
    Again, (3.6) is used to prove the assertion.
    By Part (b),
    \[
        \begin{Vmatrix}
            Ax
        \end{Vmatrix}_{2}
        \leq
        \sqrt{m}
        \begin{Vmatrix}
            Ax
        \end{Vmatrix}_{\infty}. 
        \]
        Then by Part (a)
        \[
    \frac{
        1
    }{
        \begin{Vmatrix}
            x
        \end{Vmatrix}_{2}
    } 
    \leq 
    \frac{
        1
    }{
        \begin{Vmatrix}
            x
        \end{Vmatrix}_{\infty}
    }.
    \]
    Putting it all together yields,
    \[
        \begin{Vmatrix}
            A
        \end{Vmatrix}_{2}
        =
        \sup_{x \neq 0}
        \frac{
            \begin{Vmatrix}
                Ax
            \end{Vmatrix}_{2}
        }{
            \begin{Vmatrix}
                x
            \end{Vmatrix}_{2}
        } 
        \leq 
        \sup_{x \neq 0}
        \frac{
            \sqrt{m}
            \begin{Vmatrix}
                Ax
            \end{Vmatrix}_{\infty}
        }{
            \begin{Vmatrix}
                x
            \end{Vmatrix}_{\infty}
        } 
        =
        \sqrt{m}
        \begin{Vmatrix}
            A
        \end{Vmatrix}_{\infty}.
    \]
        Therefore,   \[
            \begin{Vmatrix}
                A
            \end{Vmatrix}_{2}
            \leq
            \sqrt{m}
            \begin{Vmatrix}
                A
            \end{Vmatrix}_{\infty}. 
            \]
\end{proof}



As an example, Consider 
$A = 
\begin{bmatrix}
    1 & 0\\
    1 & 0\\
\end{bmatrix} $
 and
 $x = \begin{bmatrix}
    1\\
    0\\
 \end{bmatrix}$. Computing the $\infty$-norm for $Ax$ and $x$ yields:
 \[ 
    \begin{Vmatrix}
        x
    \end{Vmatrix}_\infty
    =
    \begin{Vmatrix}
        \begin{bmatrix}
            1\\
            0\\
        \end{bmatrix}
    \end{Vmatrix}_\infty
    = \max \{1, 0\}
    = 1,
 \]
 and
 \[
    \begin{Vmatrix}
        Ax
    \end{Vmatrix}_\infty
    =
    \begin{Vmatrix}
        \begin{bmatrix}
            1 & 0\\
            1 & 0\\
        \end{bmatrix}
        \begin{bmatrix}
            1\\
            0\\
        \end{bmatrix}
    \end{Vmatrix}_\infty
    =
    \begin{Vmatrix}
        \begin{bmatrix}
            1\\
            1\\
        \end{bmatrix}
    \end{Vmatrix}_\infty
    =
    1
 \]
 Thus, 
 \[\sqrt{2}
    \begin{Vmatrix}
        A
    \end{Vmatrix}_\infty
    =
    \sqrt{2}
    \frac{
            \begin{Vmatrix}
                Ax
            \end{Vmatrix}_{\infty}
        }{
            \begin{Vmatrix}
                x
            \end{Vmatrix}_{\infty}
        } 
    =
    \sqrt{2}\cdot
    \frac{1}{1} = \sqrt{2}.
\]
Computing the $2$-norm for $Ax$ and $x$ yields:
 \[ 
    \begin{Vmatrix}
        x
    \end{Vmatrix}_2
    =
    \begin{Vmatrix}
        \begin{bmatrix}
            1\\
            0\\
        \end{bmatrix}
    \end{Vmatrix}_2
    = \sqrt{\left(1\right)^2 + \left(0\right)^2}
    = 1,
 \]
 and
 \[
    \begin{Vmatrix}
        Ax
    \end{Vmatrix}_2
    =
    \begin{Vmatrix}
        \begin{bmatrix}
            1 & 0\\
            1 & 0\\
        \end{bmatrix}
        \begin{bmatrix}
            1\\
            0\\
        \end{bmatrix}
    \end{Vmatrix}_2
    =
    \begin{Vmatrix}
        \begin{bmatrix}
            1\\
            1\\
        \end{bmatrix}
    \end{Vmatrix}_2
    = \sqrt{1^2 + 1^2}
    = \sqrt{2}
 \]
 Thus, 
 \[ 
    \begin{Vmatrix}
        A
    \end{Vmatrix}_2
    =
    \frac{
            \begin{Vmatrix}
                Ax
            \end{Vmatrix}_{2}
        }{
            \begin{Vmatrix}
                x
            \end{Vmatrix}_{2}
        } 
    =
    \frac{\sqrt{2}}{1} = \sqrt{2}.
\]
Putting the computations together, we have 
\[
    \begin{Vmatrix}
        A
    \end{Vmatrix}_2
    =
    \sqrt{2}
    =
    \sqrt{2}
    \begin{Vmatrix}
        A
    \end{Vmatrix}_\infty
\]






\pagebreak
\section*{Problem 3.4}
Let $A$ be an $m \times n$ matrix and let $b$ be a submatrix of $A$, that is, a $\mu \times \nu$ matrix ($\mu \leq m, \nu \leq n$) obtained by selecting certain rows and columns of $A$. \\

\textbf{(a)} Explain how $B$ can be obtained by multiplying $A$ by certain row and column ``deletion matrices'' as in step 7 of Exercise 1.1. \\

\textbf{Solution:} The $\mu \times \nu$ matrix $B$ can be obtained from the $m \times n$ matrix $A$ by left multiplication by a $\mu \times m$ matrix $L$ and a $n \times \nu$ matrix $R$. To remove row $i$ from $A$, column $i$ of $L$ should have only zeros. Every column of $L$ must contain exactly one 1. To remove column $j$ from $A$, row $j$ of $R$ should have all zeros. Every row of $R$ must contain exactly one 1.\\

\textbf{(b)} Using this product, show that
$
\begin{Vmatrix}
    B
\end{Vmatrix}_p
\leq 
\begin{Vmatrix}
    A
\end{Vmatrix}_p
$
for any $p$ with $1 \leq p \leq \infty$.
\begin{proof}
    Since $L$ and $R$ have columns and rows containing at most a single 1, the $p$-norms for $L$ and $R$ are at most 1, that is
    \[
        \begin{Vmatrix}
            L
        \end{Vmatrix}_p
        \leq 1
        ,
        ~~~~
        \begin{Vmatrix}
            R
        \end{Vmatrix}_p
        \leq 1
    \]
    Then using (3.14) from page 22 of the text, we have 
    \[
        \begin{Vmatrix}
            B
        \end{Vmatrix}_p
        = 
        \begin{Vmatrix}
            LAR
        \end{Vmatrix}_p
        \leq
        \begin{Vmatrix}
            L
        \end{Vmatrix}_p
        \begin{Vmatrix}
            A
        \end{Vmatrix}_p
        \begin{Vmatrix}
            R
        \end{Vmatrix}_p
        \leq
        1 \cdot
        \begin{Vmatrix}
            A
        \end{Vmatrix}_p
        \cdot 1
        =
        \begin{Vmatrix}
            A
        \end{Vmatrix}_p.
    \]
    Therefore, $
    \begin{Vmatrix}
        B
    \end{Vmatrix}_p
    \leq 
    \begin{Vmatrix}
        A
    \end{Vmatrix}_p
    $
\end{proof}
\end{document}
