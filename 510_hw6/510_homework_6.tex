\documentclass{article}
%\usepackage{concmath}
\usepackage{fancyhdr}
\usepackage{extramarks}
\usepackage{amsmath}
\usepackage{amsthm}
\usepackage{amsfonts}
\usepackage{tikz}
\usepackage[plain]{algorithm}
\usepackage{algpseudocode}

\usetikzlibrary{automata,positioning}

%
% Basic Document Settings
%

\topmargin=-0.45in
\evensidemargin=0in
\oddsidemargin=0in
\textwidth=6.5in
\textheight=9.0in
\headsep=0.25in

\linespread{1.1}

\pagestyle{fancy}
\lhead{\hmwkAuthorName}
\chead{\hmwkClass\ : \hmwkTitle}
%\rhead{\firstxmark}
\lfoot{\lastxmark}
\cfoot{\thepage}

\renewcommand\headrulewidth{0.4pt}
\renewcommand\footrulewidth{0.4pt}

\setlength\parindent{0pt}

%
% Create Problem Sections
%

\newcommand{\enterProblemHeader}[1]{
    \nobreak\extramarks{}{Problem \arabic{#1} continued on next page\ldots}\nobreak{}
    \nobreak\extramarks{Problem \arabic{#1} (continued)}{Problem \arabic{#1} continued on next page\ldots}\nobreak{}
}

\newcommand{\exitProblemHeader}[1]{
    \nobreak\extramarks{Problem \arabic{#1} (continued)}{Problem \arabic{#1} continued on next page\ldots}\nobreak{}
    \stepcounter{#1}
    \nobreak\extramarks{Problem \arabic{#1}}{}\nobreak{}
}

\setcounter{secnumdepth}{0}
\newcounter{partCounter}
\newcounter{homeworkProblemCounter}
\setcounter{homeworkProblemCounter}{1}
\nobreak\extramarks{Problem \arabic{homeworkProblemCounter}}{}\nobreak{}

%
% Homework Problem Environment
%
% This environment takes an optional argument. When given, it will adjust the
% problem counter. This is useful for when the problems given for your
% assignment aren't sequential. See the last 3 problems of this template for an
% example.
%
\newenvironment{homeworkProblem}[1][-1]{
    \ifnum#1>0
        \setcounter{homeworkProblemCounter}{#1}
    \fi
    \section{Problem \arabic{homeworkProblemCounter}}
    \setcounter{partCounter}{1}
    \enterProblemHeader{homeworkProblemCounter}
}{
    \exitProblemHeader{homeworkProblemCounter}
}

%
% Homework Details
%   - Title
%   - Due date
%   - Class
%   - Section/Time
%   - Instructor
%   - Author
%

\newcommand{\hmwkTitle}{Homework\ \#6}
\newcommand{\hmwkDueDate}{September 14, 2022}
\newcommand{\hmwkClass}{Numerical Linear Algebra}
\newcommand{\hmwkClassTime}{Section 1}
\newcommand{\hmwkClassInstructor}{Instructor: Professor Blake Barker\\}
\newcommand{\hmwkAuthorName}{\textbf{Michael Snyder}}

%
% Title Page
%

\title{
    \vspace{2in}
    \textmd{\textbf{\hmwkClass:\ \hmwkTitle}}\\
    \normalsize\vspace{0.1in}\small{Due\ on\ \hmwkDueDate\ at 10:00PM}\\
    \vspace{0.1in}\large{\textit{\hmwkClassInstructor\ \hmwkClassTime}}
    \vspace{3in}
}

\author{\hmwkAuthorName}
\date{}

\renewcommand{\part}[1]{\textbf{\large Part \Alph{partCounter}}\stepcounter{partCounter}\\}

%
% Various Helper Commands
%

% Useful for algorithms
\newcommand{\alg}[1]{\textsc{\bfseries \footnotesize #1}}

% For derivatives
\newcommand{\deriv}[1]{\frac{\mathrm{d}}{\mathrm{d}x} (#1)}

% For partial derivatives
\newcommand{\pderiv}[2]{\frac{\partial}{\partial #1} (#2)}

% Integral dx
\newcommand{\dx}{\mathrm{d}x}

% Alias for the Solution section header
\newcommand{\solution}{\textbf{\large Solution}}

% Probability commands: Expectation, Variance, Covariance, Bias
\newcommand{\E}{\mathrm{E}}
\newcommand{\Var}{\mathrm{Var}}
\newcommand{\Cov}{\mathrm{Cov}}
\newcommand{\Bias}{\mathrm{Bias}}

\begin{document}

\maketitle

\pagebreak
\section*{Problem 6.1}
If $P$ is an orthogonal projector, then $I - 2P$ is unitary. Prove this algebraically, and give a geometric interpretation.

\begin{proof}
    Suppose $P$ is an orthogonal projector. A matrix $Q$ is unitary if $Q = Q^{-1}$, or $Q^*Q = QQ^* = I$. Using this relationship, we find
    \begin{align}
        (I - 2P)^*(I - 2P) &= I^*I - 2I^*P - 2P^*I + 2P^*(2P)\\
        &= I - 4P + 4P\\
        &= I.
    \end{align}
    This verifies that $I - 2P$ is unitary. 
\end{proof}

\textbf{Geometric Interpretation}
Suppose $P \in C^{m \times m}$ is a projector. Then $P$ partitions $C^{m \times m}$ into subspaces $\text{range}(P) = \text{null}(I-P)$ and $\text{null}(P) = \text{range}(I-P)$. Thus we consider vectors in these complementary spaces. Let the vector $v \in \text{range}(P)$. Then \[(I - 2P)v = [(I-P) - P]v = (I-P)v - Pv = -v.\] That is, $I-2P$ reflects vectors in $\text{range}(P)$. If we instead consider $w \in \text{null}(P) = \text{range}(I - P)$, then we have \[(I - 2P)w = [(I-P) - P]w = (I-P)w - Pw = w. \] Thus, $I - 2P$ is the identity for vectors in $\text{range}(I - P)$.


\pagebreak
\section*{Problem 6.3}
Given $A \in \mathbb{C}^{m \times n}$ with $m \geq n$, show that $A^*A$ is nonsingular if and only if $A$ has full rank.\\

\begin{proof}
    $(\Rightarrow)$ Suppose first that $A \in \mathbb{C}^{m \times n}$ such that $A^*A$ is nonsingular. This means that $A^*Ax = 0$ if and only if $x = 0$. That is, $\text{null}(A^*A) = \{0\}$. Suppose to the contrary that $A$ is not full rank. Then there exists $x \neq 0$ such that $y = Ax = 0$. But this means \[(A^*A)x = A^*(Ax) = A^*y = 0. \] Since $x \neq 0$, this contradicts the fact that $A^*A$ is nonsingular. Therefore, $x$ must equal zero and $A$ must have full rank.\\

    $(\Leftarrow)$ Now suppose that $A \in \mathbb{C}^{m \times n}$ has full rank. Then by Theorem 1.3, $0$ is not a singular value of $A$. By Theorem 5.3 this also means that $\lambda = 0$ is not an eigenvalue of $A^*A = V\Sigma^2V^*$. Therefore, again by Theorem 1.3, $A^*A$ is nonsingular.  

\end{proof}


\end{document}
