\documentclass{article}
%\usepackage{concmath}
\usepackage{fancyhdr}
\usepackage{extramarks}
\usepackage{amsmath}
\usepackage{amsthm}
\usepackage{amsfonts}
\usepackage{array}
\usepackage{tikz}
\usepackage[plain]{algorithm}
\usepackage{algpseudocode}

\usetikzlibrary{automata,positioning}

\newcolumntype{L}{>{$}l<{$}}

%
% Basic Document Settings
%

\topmargin=-0.45in
\evensidemargin=0in
\oddsidemargin=0in
\textwidth=6.5in
\textheight=9.0in
\headsep=0.25in

\linespread{1.1}

\pagestyle{fancy}
\lhead{\hmwkAuthorName}
\chead{\hmwkClass\ : \hmwkTitle}
%\rhead{\firstxmark}
\lfoot{\lastxmark}
\cfoot{\thepage}

\renewcommand\headrulewidth{0.4pt}
\renewcommand\footrulewidth{0.4pt}

\setlength\parindent{0pt}

%
% Create Problem Sections
%

\newcommand{\enterProblemHeader}[1]{
    \nobreak\extramarks{}{Problem \arabic{#1} continued on next page\ldots}\nobreak{}
    \nobreak\extramarks{Problem \arabic{#1} (continued)}{Problem \arabic{#1} continued on next page\ldots}\nobreak{}
}

\newcommand{\exitProblemHeader}[1]{
    \nobreak\extramarks{Problem \arabic{#1} (continued)}{Problem \arabic{#1} continued on next page\ldots}\nobreak{}
    \stepcounter{#1}
    \nobreak\extramarks{Problem \arabic{#1}}{}\nobreak{}
}

\setcounter{secnumdepth}{0}
\newcounter{partCounter}
\newcounter{homeworkProblemCounter}
\setcounter{homeworkProblemCounter}{1}
\nobreak\extramarks{Problem \arabic{homeworkProblemCounter}}{}\nobreak{}

%
% Homework Problem Environment
%
% This environment takes an optional argument. When given, it will adjust the
% problem counter. This is useful for when the problems given for your
% assignment aren't sequential. See the last 3 problems of this template for an
% example.
%
\newenvironment{homeworkProblem}[1][-1]{
    \ifnum#1>0
        \setcounter{homeworkProblemCounter}{#1}
    \fi
    \section{Problem \arabic{homeworkProblemCounter}}
    \setcounter{partCounter}{1}
    \enterProblemHeader{homeworkProblemCounter}
}{
    \exitProblemHeader{homeworkProblemCounter}
}

%
% Homework Details
%   - Title
%   - Due date
%   - Class
%   - Section/Time
%   - Instructor
%   - Author
%

\newcommand{\hmwkTitle}{Homework\ \#7}
\newcommand{\hmwkDueDate}{September 16, 2022}
\newcommand{\hmwkClass}{Numerical Linear Algebra}
\newcommand{\hmwkClassTime}{Section 1}
\newcommand{\hmwkClassInstructor}{Instructor: Professor Blake Barker\\}
\newcommand{\hmwkAuthorName}{\textbf{Michael Snyder}}

%
% Title Page
%

\title{
    \vspace{2in}
    \textmd{\textbf{\hmwkClass:\ \hmwkTitle}}\\
    \normalsize\vspace{0.1in}\small{Due\ on\ \hmwkDueDate\ at 10:00PM}\\
    \vspace{0.1in}\large{\textit{\hmwkClassInstructor\ \hmwkClassTime}}
    \vspace{3in}
}

\author{\hmwkAuthorName}
\date{}

\renewcommand{\part}[1]{\textbf{\large Part \Alph{partCounter}}\stepcounter{partCounter}\\}

%
% Various Helper Commands
%

% Useful for algorithms
\newcommand{\alg}[1]{\textsc{\bfseries \footnotesize #1}}

% For derivatives
\newcommand{\deriv}[1]{\frac{\mathrm{d}}{\mathrm{d}x} (#1)}

% For partial derivatives
\newcommand{\pderiv}[2]{\frac{\partial}{\partial #1} (#2)}

% Integral dx
\newcommand{\dx}{\mathrm{d}x}

% Alias for the Solution section header
\newcommand{\solution}{\textbf{\large Solution}}

% Probability commands: Expectation, Variance, Covariance, Bias
\newcommand{\E}{\mathrm{E}}
\newcommand{\Var}{\mathrm{Var}}
\newcommand{\Cov}{\mathrm{Cov}}
\newcommand{\Bias}{\mathrm{Bias}}

\begin{document}

\maketitle

\pagebreak
\section*{Problem 7.3}
Let $A$ be an $m \times m$ matrix, and let $a_j$ be its $j^{th}$ column. Give an algebraic proof of \textit{Hadmamard's inequality:} \[ \lvert \det A \rvert \leq \prod_{j=1}^m \begin{Vmatrix}
    a_j
\end{Vmatrix}_2. \]
Also, give a geometric interpretation of this result, making use of the fact that the determinant equals the volume of a parallelepiped. 

\begin{proof}
    Let $A$ be an $m \times m$ matrix, and let $a_j$ be its $j^{th}$ column. We will prove Hadamard's inequality using the following facts:
    \begin{enumerate}
        \item $A$ has a $QR$ factorization;
        %\item $\det QA = \det A$ for unitary matrix $Q$.
        \item For matrices $A$ and $B$, $\det AB = \det A \det B$;
        \item By Theorem 5.6, $\lvert \det A \rvert = \prod \sigma_j$. Moreover, by Theorem 5.3, $\begin{Vmatrix}
            A
        \end{Vmatrix}_2 = \sigma_1$. As a consequence of these two theorems, we have $\lvert \det Q \rvert \leq 1$;
        \item $\det U = \prod u_{ii}$ for upper triangular $U$;
        \item By algorithm 7.1, $r_{jj} = \begin{Vmatrix}
            a_j
        \end{Vmatrix}_2.$

    \end{enumerate}

    Using these facts, we have\\

    \begin{tabular}{L L L L}
        \lvert \det A \rvert & = & \lvert \det QR \rvert & ~~~ \text{Fact 1}\\
        & = & \lvert \det Q \det R \rvert & ~~~ \text{Fact 2}\\
        & \leq & \lvert \det R \rvert & ~~~ \text{Fact 3}\\
        & = & \prod_{j=1}^n r_{jj} & ~~~ \text{Fact 4}\\
        & = & \prod_{j=1}^n \begin{Vmatrix}
            a_j
        \end{Vmatrix}_2. & ~~~ \text{Fact 5}
    \end{tabular}

    % \begin{tabular}{l l l l}
    %     $\lvert \det A \rvert$ & = & $\lvert \det QR \rvert$ & ~~~ Fact 1\\
    %     & $=$ & $\lvert \det Q \det R \rvert$ & ~~~ Fact 2\\
    %     & $\leq$ & $\lvert \det R \rvert$ & ~~~ Fact 3\\
    %     & $=$ & $\prod_{j=1}^n r_{jj}$ & ~~~ Fact 4\\
    %     & $=$ & $\prod_{j=1}^n \begin{Vmatrix}
    %         a_j
    %     \end{Vmatrix}_2$. & ~~~ Fact 5
    % \end{tabular}
\end{proof}


\pagebreak
\section*{Problem 7.4}
Let $x^{(1)}, y^{(1)}, x^{(2)}, y^{(2)}$ be nonzero vectors in $\mathbb{R}^3$ with the property that $x^{(1)}$ and $y^{(1)}$ are linearly independent and so are $x^{(2)}$ and $y^{(2)}$. Consider the two planes in $\mathbb{R}^3$, \[ P^{(1)} = \langle x^{(1)}, y^{(1)} \rangle, ~~~~~ P^{(2)} = \langle x^{(2)}, y^{(2)} \rangle. \] Suppose we wish to find a nonzero vector $v \in \mathbb{R}^3$ that lies in the intersection $P =  P^{(1)} \cap P^{(2)}.$  Devise a method for solving this problem by reducing it to the computation of $QR$ factorizations of $3 \times 2$ matrices.\\

\textbf{Solution:} Let $A^{(1)} = \begin{bmatrix}
    x^{(1)} & \lvert &  y^{(1)}
\end{bmatrix}$, be the $3 \times 2$ matrix whose columns are the vectors $x^{(1)}$ and $y^{(1)}$. Similarly, form the matrix $A^{(2)} = \begin{bmatrix}
    x^{(2)} & \lvert &  y^{(2)}.
\end{bmatrix}$
Then \[ A^{(1)} = \begin{bmatrix}
    x^{(1)} & \lvert &  y^{(1)}
\end{bmatrix} = Q^{(1)}R^{(1)} = \begin{bmatrix}
    q_1^{(1)} & \lvert &  q_2^{(1)} & \lvert & q_3^{(1)}
\end{bmatrix}R^{(1)},\]
where $Q^{(1)}R^{(1)}$ is the QR factorization of $A^{(1)}$ and hence by Algorithm 7.1, $q_1^{(1)}, q_2^{(1)}, q_3^{(1)}$ are the orthonormal columns of $Q^{(1)}$. But $range (A^{(1)})$ is spanned by the vectors $q_1^{(1)}, q_2^{(1)}$. Stated another way, $span(\{ q_1^{(1)}, q_2^{(1)} \}) = P^{(1)}$. Since $q_3^{(1)}$ is orthogonal to $q_1^{(1)}$ and  $q_2^{(1)}$, $q_3^{(1)}$ is also orthogonal to $P^{(1)}$. By similar analysis, we obtain, through QR factorization of the matrix $A^{(2)}$, the vector $q_3^{(2)}$ orthogonal to $P^{2}$.\\

Now consider the matrix $A^{(3)} = \begin{bmatrix}
    q_3^{(1)} & \lvert &  q_3^{(2)}
\end{bmatrix}$. By the same process as above we find the QR factorization of $A^{(3)}$, which yields the matrix $Q^{(3)} =  \begin{bmatrix}
    q_3^{(1)} & \lvert &  q_3^{(2)} & \lvert & q_3^{(3)} \end{bmatrix}.$ Again, $q_3^{(1)}$ and $q_3^{(2)}$ are orthogonal to $q_3^{(3)}$. But that puts $q_3^{(3)}$ in the intersection of the planes $P^{(1)}$ and  $P^{(2)}$. Therefore, we have found the requested vector $v = q_3^{(3)}$ through QR factorization of $3 \times 2$ matrices.


\end{document}
