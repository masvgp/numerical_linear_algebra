\documentclass{article}
%\usepackage{concmath}
\usepackage{fancyhdr}
\usepackage{extramarks}
\usepackage{amsmath}
\usepackage{amsthm}
\usepackage{amsfonts}
\usepackage{array}
\usepackage{tikz}
\usepackage[plain]{algorithm}
\usepackage{algpseudocode}

\usetikzlibrary{automata,positioning}

\newcolumntype{L}{>{$}l<{$}}

%
% Basic Document Settings
%

\topmargin=-0.45in
\evensidemargin=0in
\oddsidemargin=0in
\textwidth=6.5in
\textheight=9.0in
\headsep=0.25in

\linespread{1.1}

\pagestyle{fancy}
\lhead{\hmwkAuthorName}
\chead{\hmwkClass\ : \hmwkTitle}
%\rhead{\firstxmark}
\lfoot{\lastxmark}
\cfoot{\thepage}

\renewcommand\headrulewidth{0.4pt}
\renewcommand\footrulewidth{0.4pt}

\setlength\parindent{0pt}

%
% Create Problem Sections
%

\newcommand{\enterProblemHeader}[1]{
    \nobreak\extramarks{}{Problem \arabic{#1} continued on next page\ldots}\nobreak{}
    \nobreak\extramarks{Problem \arabic{#1} (continued)}{Problem \arabic{#1} continued on next page\ldots}\nobreak{}
}

\newcommand{\exitProblemHeader}[1]{
    \nobreak\extramarks{Problem \arabic{#1} (continued)}{Problem \arabic{#1} continued on next page\ldots}\nobreak{}
    \stepcounter{#1}
    \nobreak\extramarks{Problem \arabic{#1}}{}\nobreak{}
}

\setcounter{secnumdepth}{0}
\newcounter{partCounter}
\newcounter{homeworkProblemCounter}
\setcounter{homeworkProblemCounter}{1}
\nobreak\extramarks{Problem \arabic{homeworkProblemCounter}}{}\nobreak{}

%
% Homework Problem Environment
%
% This environment takes an optional argument. When given, it will adjust the
% problem counter. This is useful for when the problems given for your
% assignment aren't sequential. See the last 3 problems of this template for an
% example.
%
\newenvironment{homeworkProblem}[1][-1]{
    \ifnum#1>0
        \setcounter{homeworkProblemCounter}{#1}
    \fi
    \section{Problem \arabic{homeworkProblemCounter}}
    \setcounter{partCounter}{1}
    \enterProblemHeader{homeworkProblemCounter}
}{
    \exitProblemHeader{homeworkProblemCounter}
}

%
% Homework Details
%   - Title
%   - Due date
%   - Class
%   - Section/Time
%   - Instructor
%   - Author
%

\newcommand{\hmwkTitle}{Homework\ \#8}
\newcommand{\hmwkDueDate}{September 19, 2022}
\newcommand{\hmwkClass}{Numerical Linear Algebra}
\newcommand{\hmwkClassTime}{Section 1}
\newcommand{\hmwkClassInstructor}{Instructor: Professor Blake Barker\\}
\newcommand{\hmwkAuthorName}{\textbf{Michael Snyder}}

%
% Title Page
%

\title{
    \vspace{2in}
    \textmd{\textbf{\hmwkClass:\ \hmwkTitle}}\\
    \normalsize\vspace{0.1in}\small{Due\ on\ \hmwkDueDate\ at 10:00PM}\\
    \vspace{0.1in}\large{\textit{\hmwkClassInstructor\ \hmwkClassTime}}
    \vspace{3in}
}

\author{\hmwkAuthorName}
\date{}

\renewcommand{\part}[1]{\textbf{\large Part \Alph{partCounter}}\stepcounter{partCounter}\\}

%
% Various Helper Commands
%

% Useful for algorithms
\newcommand{\alg}[1]{\textsc{\bfseries \footnotesize #1}}

% For derivatives
\newcommand{\deriv}[1]{\frac{\mathrm{d}}{\mathrm{d}x} (#1)}

% For partial derivatives
\newcommand{\pderiv}[2]{\frac{\partial}{\partial #1} (#2)}

% Integral dx
\newcommand{\dx}{\mathrm{d}x}

% Alias for the Solution section header
\newcommand{\solution}{\textbf{\large Solution}}

% Probability commands: Expectation, Variance, Covariance, Bias
\newcommand{\E}{\mathrm{E}}
\newcommand{\Var}{\mathrm{Var}}
\newcommand{\Cov}{\mathrm{Cov}}
\newcommand{\Bias}{\mathrm{Bias}}

\begin{document}

\maketitle

\pagebreak
\section*{Problem 8.1}
Let $A$ be an $m \times n$ matrix. Determine the exact number of floating point operations (i.e., additions, subtractions, multiplications, and divisions) involved in computing the factorization $A = \hat{Q}\hat{R}$ by Algorithm 8.1. \\

\textbf{Solution:} First we note that because we are considering the reduced QR factorization, $Q$ is $m \times n$ and $R$ is $n \times n$. With this in mind, and beginning from the inner loop, we find that 
\subsection*{Inner Loop}
$r_{ij}q_i$ results $m$ multiplications and $v_j - r_{ij}q_i$ results in $m$ subtractions. Thus, $v_j = v_j - r_{ij}q_i$ results in $2m$ floating point operations.\\

$q_i^*v_j$ results in $m$ multiplications and $m-1$ additions. Thus, $r_{ij}=q_i^*v_j$ results in $m + m-1 = 2m - 1$ floating point operations.\\

Adding these two assignments within the inner loop together, we have $4m-1$ floating point operations.\\

Since these assignments happen $\text{for } i=1 \text{ to } n$ and for $j=i+1$ to $n$, we are summing up the $4m-1$ floating point operations $n$ times starting at $j=1+1=2$ to $n$ which means we have $\frac{n}{2}(n - (i + 1) + 1) = \frac{n}{2}(n - 2 + 1) = \frac{n(n-1)}{2}$ terms. Thus we have \[\sum_{i=1}^n\sum_{j=i+1}^n(4m-1) = \frac{1}{2}\cdot n(n-1) \cdot (4m-1) =  \frac{n(n-1)}{2}(4m-1)\] floating point operations.

\subsection*{Outer Loop}
The outer loop consists of $m$ divisions in $v_i/r_{ii}$ and $m$ multiplications and $m-1$ additions in $\begin{Vmatrix}
    v_i
\end{Vmatrix}.$

Then summing those up for $i=1$ to $n$ we have $n(3m-1)$ floating point operations. \\

We note here prior to summing all the floating point operations that the top loop of Algorithm 8.1 (i.e., with $v_i = a_i$) does not result in any floating point operations since it is only reassignment.\\

Now, adding up the inner and outer loops, we have \[\text{inner loop } + \text{ outer loop } = \frac{n(n-1)}{2}(4m-1) + n(3m-1)\] floating point operations.



\end{document}
